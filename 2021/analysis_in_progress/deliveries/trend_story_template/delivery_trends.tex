% Options for packages loaded elsewhere
\PassOptionsToPackage{unicode}{hyperref}
\PassOptionsToPackage{hyphens}{url}
%
\documentclass[
  12pt,
]{article}
\usepackage{amsmath,amssymb}
\usepackage{lmodern}
\usepackage{iftex}
\ifPDFTeX
  \usepackage[T1]{fontenc}
  \usepackage[utf8]{inputenc}
  \usepackage{textcomp} % provide euro and other symbols
\else % if luatex or xetex
  \usepackage{unicode-math}
  \defaultfontfeatures{Scale=MatchLowercase}
  \defaultfontfeatures[\rmfamily]{Ligatures=TeX,Scale=1}
\fi
% Use upquote if available, for straight quotes in verbatim environments
\IfFileExists{upquote.sty}{\usepackage{upquote}}{}
\IfFileExists{microtype.sty}{% use microtype if available
  \usepackage[]{microtype}
  \UseMicrotypeSet[protrusion]{basicmath} % disable protrusion for tt fonts
}{}
\makeatletter
\@ifundefined{KOMAClassName}{% if non-KOMA class
  \IfFileExists{parskip.sty}{%
    \usepackage{parskip}
  }{% else
    \setlength{\parindent}{0pt}
    \setlength{\parskip}{6pt plus 2pt minus 1pt}}
}{% if KOMA class
  \KOMAoptions{parskip=half}}
\makeatother
\usepackage{xcolor}
\usepackage[margin=1in]{geometry}
\usepackage{graphicx}
\makeatletter
\def\maxwidth{\ifdim\Gin@nat@width>\linewidth\linewidth\else\Gin@nat@width\fi}
\def\maxheight{\ifdim\Gin@nat@height>\textheight\textheight\else\Gin@nat@height\fi}
\makeatother
% Scale images if necessary, so that they will not overflow the page
% margins by default, and it is still possible to overwrite the defaults
% using explicit options in \includegraphics[width, height, ...]{}
\setkeys{Gin}{width=\maxwidth,height=\maxheight,keepaspectratio}
% Set default figure placement to htbp
\makeatletter
\def\fps@figure{htbp}
\makeatother
\setlength{\emergencystretch}{3em} % prevent overfull lines
\providecommand{\tightlist}{%
  \setlength{\itemsep}{0pt}\setlength{\parskip}{0pt}}
\setcounter{secnumdepth}{-\maxdimen} % remove section numbering
\usepackage{xcolor}
\usepackage{hyperref}
\usepackage{pdfcomment}
\usepackage{fancyhdr} \pagestyle{fancy} \setlength{\headheight}{75pt} \setlength{\textheight}{600pt} \fancyhead[C]{} \fancyhead[L]{\includegraphics{X:/DSA/shiny-uploads/images/PST_Equity_Edition-Trend_header.png}} \fancyhead[R]{} \fancyfoot[L]{\scriptsize{1011 Western Ave, Suite 500, Seattle WA 98104} \textcolor[HTML]{F05A28}. 206.464.7532 \textcolor[HTML]{F05A28}. www.psrc.org \textcolor[HTML]{F05A28}. January 2023} \fancyfoot[R]{\textcolor[HTML]{F05A28}\thepage} \fancyfoot[C]{} \renewcommand{\headrulewidth}{0pt} \renewcommand{\footrulewidth}{4pt} \renewcommand{\footrule}{\hbox to \headwidth{\color[HTML]{BCBEC0}\leaders\hrule height \footrulewidth\hfill}}
\usepackage{fontspec}
\ifLuaTeX
  \usepackage{selnolig}  % disable illegal ligatures
\fi
\IfFileExists{bookmark.sty}{\usepackage{bookmark}}{\usepackage{hyperref}}
\IfFileExists{xurl.sty}{\usepackage{xurl}}{} % add URL line breaks if available
\urlstyle{same} % disable monospaced font for URLs
\hypersetup{
  pdfauthor={Meg Grzybowski},
  hidelinks,
  pdfcreator={LaTeX via pandoc}}

\author{Meg Grzybowski}
\date{2022-12-29}

\begin{document}

\setmainfont{Poppins}

\hypertarget{delivery-trends-in-the-puget-sound-region-2017-2019-2021}{%
\section{Delivery Trends in the Puget Sound Region (2017, 2019,
2021)}\label{delivery-trends-in-the-puget-sound-region-2017-2019-2021}}

\begin{flushleft}
The 2021 regional travel survey collected day-to-day information from households in the central Puget Sound region: how we traveled, where we went, how long it took - even where we chose to live and whether we got home deliveries. This report compares household delivery choices in 2021, during COVID-19 conditions, to that in the previous years of 2017 and 2019. Learn more at the \href{https://www.psrc.org/our-work/household-travel-survey-program}{\underline{\textcolor{blue}{PSRC household travel survey webpage}}}. You can also \href{https://household-travel-survey-psregcncl.hub.arcgis.com}{\underline{\textcolor{blue}{view the full travel survey dataset here}}}, including 2017, 2019, and 2021 data.

## Food and Grocery Deliveries more than doubled from 2019 to 2021
\begin{flushleft}
In 2021, although the average food/meal or grocery delivery share was only 4 or 5\%, 
that represents twice the share of these type of deliveries in both 2017 and 2019.

Package deliveries, which represent the highest share of deliveries on an average weekday
for households, had a growth rate that doubles from 2019 to 2021.

Work or service deliveries remained consistent at around 5\%.

When combined visually, it's easy to see the distribution of home deliveries on an average weekday. 
\end{flushleft}

\includegraphics{delivery_trends_files/figure-latex/individual delivery type-1.pdf}
\includegraphics{delivery_trends_files/figure-latex/individual delivery type-2.pdf}
\includegraphics{delivery_trends_files/figure-latex/individual delivery type-3.pdf}
\includegraphics{delivery_trends_files/figure-latex/individual delivery type-4.pdf}
\includegraphics{delivery_trends_files/figure-latex/individual delivery type-5.pdf}

\hypertarget{foodmeal-and-package-deliveries-in-low-and-high-income-households-differed.}{%
\subsection{Food/Meal and Package Deliveries in Low and High-Income
Households
Differed.}\label{foodmeal-and-package-deliveries-in-low-and-high-income-households-differed.}}

\begin{flushleft}
Higher income households were substantially more likely to get a good or meal delivery,
as compared to lower income households in 2021, compared to 2017 and 2019.

Food or meal deliveries more than tripled between 2021 and previous years (1.5\% to 5.5\%).

While both lower income and higher income households had an increase in package deliveries,
those households with less than \textdollar 75,000 had a significant spike, growing from
about 18\% to over 30\% in 2021.

\end{flushleft}

\includegraphics{delivery_trends_files/figure-latex/income blocks-1.pdf}
\includegraphics{delivery_trends_files/figure-latex/income blocks-2.pdf}

\hypertarget{households-with-children-received-the-most-packages-in-2021.}{%
\subsection{Households with Children Received the Most Packages in
2021.}\label{households-with-children-received-the-most-packages-in-2021.}}

\begin{flushleft}
Food or meal, grocery, and work or service deliveries stayed relatively stable regardless of household composition
or year. However, package deliveries were the highest for households with kids (45\%), and then decreased as household age increased (about 30\%).
\end{flushleft}

\includegraphics{delivery_trends_files/figure-latex/lifecycle-1.pdf}

\hypertarget{food-and-grocery-deliveries-doubled-outside-of-rgcs-in-2021.}{%
\subsection{Food and Grocery Deliveries Doubled Outside of RGCs in
2021.}\label{food-and-grocery-deliveries-doubled-outside-of-rgcs-in-2021.}}

\begin{flushleft}
In 2021, the average food/meal or grocery delivery share more than doubled in households
outside of the Regional Growth Centers (RGCs) from nearly 2\% to over 4\%.

Package deliveries increased for both RGC and non-RGC households, but package deliveries increased
more significantly in RGCs, from less than 20\% in 2017 and 2019 to almost 20\% in 2021.

Work or servicedeliveries remained relatively stable from 2017 and 2019 to 2021 in both RGCs and non-RGC households, 
although non-RGC households tended to receive more average work or service deliveries than RGC households.
\end{flushleft}

\includegraphics{delivery_trends_files/figure-latex/rgc-1.pdf}
\includegraphics{delivery_trends_files/figure-latex/rgc-2.pdf}
\includegraphics{delivery_trends_files/figure-latex/rgc-3.pdf}

\hypertarget{household-deliveries-and-household-size}{%
\subsection{Household Deliveries and Household
Size}\label{household-deliveries-and-household-size}}

\begin{flushleft}
Still interpreting data here.
\end{flushleft}

\includegraphics{delivery_trends_files/figure-latex/hhsize-1.pdf}
\includegraphics{delivery_trends_files/figure-latex/hhsize-2.pdf}
\includegraphics{delivery_trends_files/figure-latex/hhsize-3.pdf}
\includegraphics{delivery_trends_files/figure-latex/hhsize-4.pdf}

\end{document}
